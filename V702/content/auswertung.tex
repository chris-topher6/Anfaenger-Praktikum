\section{Auswertung}
\label{sec:Auswertung}
  \subsection{Angabe der Messdaten}
    Im Folgenden sind die Messergebnisse des Versuches tabellarisch dargestellt.
    \begin{table}[H]
      \centering
      \caption{Die Messung des Zerfalls von Vanadium.}
      \label{tab:Vanadium1}
      \sisetup{table-format = 4.0}
      \begin{tabular}{S S}
        \toprule
        {$t [s]$} & {$N [Imp]$} \\
        \midrule
        30	& 189  \\
        60	& 197 \\
        90	& 150 \\
        120	& 159 \\
        150	& 155 \\
        180	& 132 \\
        210	& 117 \\
        240	& 107 \\
        270	& 94 \\
        300	& 100 \\
        330	& 79 \\
        360	& 69 \\
        390	& 81 \\
        420	& 46 \\
        450	& 49 \\
        480	& 61 \\
        510	& 56 \\
        540	& 40 \\
        570	& 45 \\
        600	& 32 \\
        630	& 27 \\
        660	& 43 \\
        690	& 35 \\
        720	& 19 \\
        750	& 28 \\
        780	& 27 \\
        810	& 36 \\
        840	& 25 \\
        870	& 29 \\
        900	& 18 \\
        930	& 17 \\
        960	& 24 \\
        990	& 21 \\
       1020 &	25  \\
       1050 &	21 \\
       1080 &	24 \\
       1110 &	25 \\
        1140 &	17 \\
        1170 &	20 \\
        1200 &	19 \\
        1230 &	20 \\
        1260 &	18 \\
        1290 &	16 \\
        1320 &	17 \\
      \bottomrule
    \end{tabular}
  \end{table}
  \begin{table}[H]
    \centering
    \caption{Die Messung des Zerfalls von Rhodium.}
    \label{tab:Rhodium1}
    \sisetup{table-format = 3.0}
    \begin{tabular}{S S}
      \toprule
      {$t [s]$} & {$N [Imp]$} \\
      \midrule
      15	& 667 \\
      30	& 585 \\
      45	& 474 \\
      60	& 399 \\
      75	& 304 \\
      90	& 253 \\
      105	& 213 \\
      120	& 173 \\
      135	& 152 \\
      150	& 126 \\
      165	& 111 \\
      180	&  92 \\
      195	&  79 \\
      210	&  74 \\
      225	&  60 \\
      240	&  52 \\
      255	&  56 \\
      270	&  53 \\
      285	&  41 \\
      300	&  36 \\
      315	&  37 \\
      330	&  32 \\
      345	&  36 \\
      360	&  38 \\
      375	&  34 \\
      390	&  40 \\
      405	&  21 \\
      420	&  35 \\
      435	&  33 \\
      450	&  36 \\
      465	&  20 \\
      480	&  24 \\
      495	&  30 \\
      510	&  30 \\
      525	&  26 \\
      540	&  28 \\
      555	&  23 \\
      570	&  20 \\
      585	&  28 \\
      600	&  17 \\
      615	&  26 \\
      630	&  19 \\
      645	&  13 \\
      660	&  17 \\
      \bottomrule
    \end{tabular}
  \end{table}
  \subsection{Bestimmung der Untergrundrate}
  Die Untergrundrate $N_{U}$ wurde mehrfach mit einem Messintervall von $t = \SI{300}{\second}$ gemessen. Im Folgenden ist $N_{U}$ tabellarisch dargestellt.
  \begin{table}
    \centering
    \caption{Die gemessene Untergrundrate $N_{U}$.}
    \label{tab:untergrundrate1}
    \sisetup{table-format=3.0}
    \begin{tabular}{S S}
      \toprule
      {$t [s]$} & {$N [Imp]$} \\
      \midrule
      0 & 129 \\
      300 & 143 \\
      600 & 144 \\
      900 & 136 \\
      1200 & 139 \\
      1500 & 126 \\
      1800 & 158 \\
      \bottomrule
    \end{tabular}
  \end{table}
  Aus diesen Werten für den Nulleffekt wurde mithilfe von Python der Mittelwert berechnet. Die Messzeit wird mit einer Ungenauigkeit von $\increment t = 10^{-5}$ aufgenommen.
  \begin{equation}
    \label{eqn:nulleffekt1}
    \bar{N_{U}} = 134 \pm 4 \si{Imp}
  \end{equation}
  Zusätzlich wird die Untergrundrate \eqref{eqn:nulleffekt1} an die Messzeiten von Vanadium ($t=\SI{30}{\second}$) und Rhodium ($t=\SI{15}{\second}$) angepasst.
\subsection{Bestimmung der Halbwertszeit von Vanadium}
  Zur Bestimmung der Halbwertszeit des Vanadiums wurde von den gemessenen Werten der Nulleffekt abgezogen. Aufgrund der Poisson-Verteilung der Werte lässt sich die Messunsicherheit nach
  Gleichung \eqref{eqn:messunsvanadium} berechnen.
  \begin{equation}
    \label{eqn:messunsvanadium}
    \increment N = \sqrt{N}
  \end{equation}
  Aus den so erhaltenen Werten und Unsicherheiten ergibt sich der im Folgenden dargestellte Plot \ref{fig:PlotVanadiumLinLog1} mit halblogarithmischer Darstellung.
  \begin{figure}[H]
    \centering
    \label{fig:PlotVanadiumLinLog1}
    \includegraphics[scale=0.7]{auswertung/PlotVanadiumLinLog1.pdf}
  \end{figure}
  An die Messwerte wurde eine lineare Regression angenähert, woran die Zerfallszeit nach dem Zerfallsgesetz \ref{eqn:zerfallsgesetz} bestimmt wurde.
