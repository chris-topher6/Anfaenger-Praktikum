\section{Diskussion}
\label{sec:Diskussion}
Die Ergebnisse aus Kapitel \ref{sec:Auswertung} sind im Folgenden noch einmal zusammengefasst.
\begin{table}[H]
    \centering
        \caption{Photonenergie bei $K_{\alpha}$ und $K_{\beta}$}
        \label{tab:energiediss}
        \sisetup{table-format=4.1}
        \begin{tabular}{S S S S}
          \toprule
          {Spektrallinie} & {Energie $[\si{\kilo\electronvolt}]$} & {Literaturwert \cite{AP03} $[\si{\kilo\electronvolt}]$} & {Abweichung [\%]}\\
          \midrule
          {$K_{\alpha}$} & 8.044 & 8.048 & 0.0472 \\
          {$K_{\beta} $} & 8.915 & 8.907 & 0.091 \\
          \bottomrule
        \end{tabular}
      \end{table}

\begin{table}[H]
        \centering
            \caption{Die Compton-Wellenlänge des Elektrons}
            \label{tab:comptondiss}
            \sisetup{table-format=1.4}
            \begin{tabular}{S @{${}\pm{}$} S S S}
              \toprule
              \multicolumn{2}{c}{$\lambda_c [\si{\pico\metre}]$} & {$\lambda_{Lit}$ \cite{AP04} $[\si{\pico\metre}]$} & {Abweichung [\%]}\\
              \midrule
              3.7593 & 0.0592 & 2.2426 & 67.63\\
              \bottomrule
            \end{tabular}
          \end{table}
\noindent
Die brechneten Energien der $K_{\alpha}$ und $K_{\beta}$ \ref{tab:energiediss} gleichen den Literaturwerten in sehr hohem
Maße, was auf eine präzise Messung schließen lässt. \\
Auch die Compton-Wellenlänge des Elektrons stimmt von der Größenordnung mit dem Literaturwert überein, wobei die Abweichung 
wesentlich größer ist, als bei bei den Spektrallinien. 
\\\noindent
Durch die geringe Unsicherheit der gemessenen Compton-Wellenlänge ist eher von einem systematischen als von einem 
statistischen Fehler auszugehen. Ein mögicher Grund für die Abweichung besteht in der Tatsache, dass 
die berechnete Compton-Wellenlänge sehr klein ist. So führen kleine Ungenauigkeiten, sowohl systematischer als auch statistischer
Natur, bei der Messung zu großen Fehlern bei dem Ergebnis. Diese Ungenauigkeiten können sowohl bei dem Aufnehmen der Daten, als 
auch durch die verwendeten Materialien, wie dem LiF-Kristall, aufgetreten sein. Eine weitere Fehlerquelle könnte in einer 
unzureichenden Abschirmung des Geiger-Müller-Zählrohrs von äußerer Strahlung sein.