\section{Diskussion}
\label{sec:Diskussion}
Die Ergebnisse aus \ref{sec:Auswertung} sind im Folgenden zusammengefasst:
\begin{align*}
  A_{0,1} & = 1983280.617 \pm 2471197.372 \si{\ampere\per\meter}\\
  b_1     & = 8.301 \cdot 10^{-5} \pm 6.230 \cdot 10^{-7} \si{\milli\meter}\\
  A_{0,2} & =  1983280.617 \pm 2471197.372 \si{\ampere\per\meter}\\
  b_2     & = 0.00104 \pm 1.052 \cdot 10^{-8} \si{\milli\meter}.
\end{align*}

\begin{itemize}
  \item \textit{Der Fit} \\
    Die Abweichungen der Berechnung der Spaltbreite sind mit $44.7 \si{\percent}$
    sowie $4.02 \si{\percent}$ nicht besonders hoch. Trotzdem ähneln die gefitteten
    Funktionen den nach der Theorie erwarteten eher nicht. Das liegt an der sehr
    komplizierten Funktion, bei derer \textit{Scientific Python} und die Methode
    \textit{curve_fit} nur sehr eingeschränkt funktioniert. Ohne Angabe eines
    Intervalles für die Parameter ergibt sich kein sinnvoller Fit, und auch die
    angegebenen Intervalle sorgen nur für mäßige Ergebnisse. Problematisch ist ebenso,
    dass hier die Qualität des Fits von der Geschicklichkeit des Experimentierenden
    abhängt, sodass hier nicht ausgeschlossen werden kann, dass die Wahl der Intervalle
    nicht ideal ist.

  \item \textit{Die Umgebung}
    Bei der Aufnahme der Messwerte konnte keine vollständige Dunkelheit garantiert
    werden, da das Ablesen und Notieren der Messwerte zwingend eine Lichtquelle
    benötigte. Dies ließe sich durch eine digitale Aufnahme der Daten beheben.
    Alternativ könnte der Zeiger des Amperemeters fluoreszierend sein, was zumindest
    die Lichtquelle in der Nähe des Detektors eliminieren würde. Auch ist die
    Abdunkelung des Raumes nur begrenzt erfolgreich.

  \item \textit{Die Messgenauigkeit}
    Die Skala des Amperemeters hat aufgrund ihrer analogen Beschaffenheit eine
    begrenzte Genauigkeit. Dies könnte mit einer digitalen Anzeige gelöst werden.
    Auch die Messung der Abstände zwischen dem Detektor, der Blende und dem Laser
    sind mit einer gewissen Ungenauigkeit versehen. Außerdem war unklar, wo genau
    der Laser in dem Gehäuse montiert ist, wodurch die Messung der Abstände
    möglicherweise einen weiteren Fehler aufweisen. Auch das händische Verstellen
    des Detektors könnte weitere Unsicherheiten erzeugt haben.


\begin{itemize}
