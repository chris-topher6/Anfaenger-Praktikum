\section{Theorie}
\label{sec:Theorie}
\subsection{Grundlagen}
\label{sec:grundlagen}
Beugung tritt immer dann auf, wenn Licht auf ein Hindernis bzw. ein Öffnung trifft, welche von den Abmessungen die Größenordnung der Wellenlänge $\lambda$ des Lichts
besitzt. Die Beugung kann dabei nicht mehr wie Rexlexion oder Brechnung über die Strahlenoptik beschrieben werden, sondern wird als Wellenvorgang
aufgefasst.
\\\noindent
Elementar für die Beschreibung des Lichts ist dabei das Huygenssche Prinzip. Dieses besagt, dass jeder Punkt der Wellenfront wieder als Elemantarwelle
aufgefasst werden kann. Durch die Superposition dieser Elementarwellen ergibt sich dann die neue Wellenfront. Aus diesem Prinzip werden in den nachfolgenden
Kapiteln die Beugungsbilder der Fernfeldbeugung (Frauenhoferbeugung) und der Nahfeldbeugung (Fresnelbeugung) hergeleitet. 

\subsection{Beugung am Einzelspalt}
\label{sec:einzel}
Trifft eine Wellenfront auf ein Hindernis mit einem Spalt, so kann nach dem Huygensschen Prinzip jeder Punkt der Spaltöffnung als Quelle für eine Elemantarwelle
aufgefasst werden. Diese Wellen breiten sich dann kugelförmig aus und interferieren mit einander. Es wird bei der Beugung zwischen zwei Näherungen unterschieden, 
die nun weiter erläutert werden.

\subsubsection*{Frauenhoferbeugung}
Die Frauenhoferbeugung nähert die Entfernung der Lichtquelle von dem Spalt als unendlich, weswegen die Lichtstrahlen alle prallel nebeneinander verlaufen, also 
eine ebene Wellenfront bilden. Somit werden alle Strahlen unter dem selben Winkel $\phi$ gebeugt. In Abbildung \ref{fig:frauenhofer} ist der Verlauf der 
Lichtstrahlen skizziert.
\begin{figure}[H]
    \centering
    \includegraphics[scale = 0.45]{pictures/frauenhofer.png}
    \caption{Frauenhoferbeugung am Einzelspalt. \cite{AP02}}
    \label{fig:frauenhofer}
\end{figure}
\noindent

\subsubsection*{Fresnelbeugung}
Bei der Fresnelbeugung ist der Abstand der Lichtquelle zu dem Spalt endlich. Die Wellenfront ist also nicht eben, sondern kugelförmig. Die Lichtstrahlen sind also
nicht parallel und werden unter unterschiedlichen Winkeln gebeugt. Dadurch ist die Fresnelbeugung mathematisch deutlich anspruchsvoller als die Frauenhoferbeugung.
In Abbildung \ref{fig:fresnel} ist die Fresnelbeugung noch einmal schematisch dargestellt.
\begin{figure}[H]
    \centering
    \includegraphics[scale = 0.45]{pictures/fresnel.png}
    \caption{Fresnelbeugung am Einzelspalt. \cite{AP02}}
    \label{fig:fresnel}
\end{figure}
\noindent

\subsubsection*{Berechnung der Intensitätsverteilung}
\label{sec:intensität}
Da die Frauenhoferbeugung aufgrund der gleichen Beugungswinkel mathematisch einfacher ist, wird im Folgenden die Intensitätsverteilung $I(\phi)$ 
nur für die Frauenhoferbeugung hergeleitet. Zudem wird ein Spalt betrachtet, dessen Breite $b$ sehr viel kleiner als seine Länge ist. Somit findet
die Beugung nur in einer Dimension statt.
\\\noindent
Wie bereits beschrieben lassen sich die einfallenden Wellen als ebene Wellen nähern. Die Amplitude der Welle ist gegeben durch 
\begin{equation*}
    A(z,t)=A_0\exp{(i(\omega t-2\pi z/\lambda))}    ,
    \label{eqn:ebenewelle}
\end{equation*}
wobei das Koordinatensystem entsprechend zu Abbildung \ref{fig:frauenhofer2} gelegt wurde. Weiter in der Abbildung zu sehen ist, dass zwei 
Lichtstrahlen im Abstand $x$ von einander, einen Wegunterschied $s$ besitzen. Dieser führ zu einem Phasenunterschied $\delta$, der durch 
\begin{equation*}
    \delta=\frac{2\pi s}{\lambda}=\frac{2\pi x \sin{\phi}}{\lambda}
    \label{eqn:phase}
\end{equation*} 
beschrieben werden kann. 

\begin{figure}[H]
    \centering
    \includegraphics[scale = 0.45]{pictures/frauenhofer2.png}
    \caption{Skizze zum Gangunterschied zweier Lichtstrahlen bei derFrauenhoferbeugung. \cite{AP01}}
    \label{fig:frauenhofer2}
\end{figure}

\noindent
Nach dem Huygensschen Prinzip kann nun die Amplitude $B(z,t,\phi)$ bestimmt werden, indem über alle Elemantarwellen 
in der Spaltenöffnung summiert wird. Wegen der infinitisimalen Breite der Strahlen und der kontinuierlichen Verteilung geht die Summe in ein 
Integral über die Spaltöffnung über
\begin{equation*}
    B(z,t,\phi)=A_0\int_0^b{\exp{\left(i\left(\omega t-\frac{2\pi z}{\lambda}+\delta\right)\right)}\symup{dx}}    .
\end{equation*}
Dieses Integral lässt sich lösen zu 
\begin{equation*}
    B(z,t,\phi) = A_0\exp{\left(i\left(\omega t-\frac{2 \pi z}{\lambda}\right)\right)}
                \cdot\exp{\left(\frac{\pi \symup{i} b \sin{(\pi)}}{\lambda}\right)}
                \cdot\frac{\sin{\eta}}{\eta}
\end{equation*}
mit
\begin{equation*}
    \eta=\frac{\pi b \sin{\phi}}{\lambda}   .
\end{equation*}
Die beiden Exponentialfunktionen sind aufgrund ihrer Komplexwertigkeit bei dem Experiment nicht von Belang, da ohnehin nur
die Intensität $I(\phi)\propto |B(z,t,\phi)|^2$ gemessen werden kann. Da der Betrag einer komplexen Exponentialfunktion $\num{1}$ ist, muss 
demnach nurnoch 
\begin{equation}
    B(\phi)=A_0b\sinc{\eta}
    \label{eqn:B}
\end{equation}
betrachtet werden. Die Nullstellen der Funktion \eqref{eqn:B} liegen bei $\pm n\lambda/b$ mit $n \in \mathbb{N}$.
Wie bereits angedeutet gilt 
\begin{equation}
    I(\phi)\propto B(\phi)^2=
    A_0b^2\left(\frac{\lambda}{\pi b \sin{\phi}}\right)^2 \cdot \sin^2{\left(\frac{\pi b \sin{\phi}}{\lambda}\right)}
    \geq 0  .
    \label{eqn:intensität1}
\end{equation}
Somit liegen bei den Nullstellen der Amplitudenfunktion \ref{eqn:B} Minima in der Intensität $I(\phi)$ vor. Die Höhe der Maxima fällt in 
Näherung mit $\phi^2$ ab.   

\begin{figure}[H]
    \centering
    \includegraphics[scale = 1.0]{pictures/intensität1.png}
    \caption{Theoriekurve der Intensitätsverteilung, wobei $\alpha=\phi$ und $d=b$. \cite{AP03}}
    \label{fig:frauenhofer2}
\end{figure}

\subsection{Beugung am Doppelspalt}
\label{sec:doppel}
\subsection{Berechnung des Intensitätsverteilung durch Fouriertransformation}
\label{sec:fourier}