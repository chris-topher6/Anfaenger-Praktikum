\section{Diskussion}
\label{sec:Diskussion}
Die Ergebnisse aus dem Abschnitt \ref{sec:Auswertung} sind in folgenden Tabellen zusammengefasst.
%Tabellen und so
%...
Die in Aufgabenteil 1-4 berechneten Werte erscheinen plausibel und entsprechen im Rahmen der Messunsicherheiten
den Literaturwerten. Bei Aufgabe 5 jedoch ist die Abweichung von der angegebenen Wellenlänge des Läsers größer,
wobei die berechneten Werte von der Größenordnung mit dem idealen Wert übereinstimmen.
Es gibt mehrere Faktoren, die die Messung beeinflusst haben können.
\begin{itemize}
    \item \textit{Ablesegenauigkeit} 
    \\\noindent
        Die Ablesegenauigkeit wurde stark durch die relativ große Skalierung der Winkelskala beschränkt.
        So kann je nach Versuch die Ablesegenauigkeit auf $\num{0.5}°$ bis $\num{1}°$ geschätzt werden.
        Die Ablesegenauigkeit wurde jedoch das Verrutschen der Vorlage zusätzlich verschlechtert, sodass
        auch systematische Fehler möglich sind. Dies kann unter anderem bei der Messenung der Beugungsmaxima 
        zu großen Ungenauigkeiten geführt haben, da die Halterung der Vorlage schlecht fixiert werden konnte.
        Damit könnte auch die relativ große Abweichung der berechneten von der angegebenen Wellenlänge
        erkären werden (vgl Tabelle \ref{tab:dis5}). 
    \item \textit{Unzureichende Vorlagengröße}
    \\\noindent
        Bei der Messung der Beugungsmaxima trat zusätzlich das Problem auf, dass viele Maxima die Vorlage 
        gar nicht erst getroffen haben, da diese zu klein war. Bei dem 300er- und 600er-Gitter führte das dazu, 
        dass nur zwei von null verschiedene Maxima gemessen werden konnten. 
    \item \textit{Ausrichtung des Lasers}
    \\\noindent
        Eine weitere Mögliche Fehlerquelle ist die Ausrichtung des Lasers, auf die Mitte der Apparatur. Trifft
        der Laser leicht versetzt auf, führt dies zu einem systematischen Messfehler, der alle Ergebnisse
        beeinflusst. Da bei der Überprüfung des Reflexionsgesetzes keine systematischen Fehler ersichtlich sind,
        scheint der Laser die Mitte der Apparaturrecht gut zu treffen. 
\end{itemize}
