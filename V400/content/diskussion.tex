\section{Diskussion}
\label{sec:Diskussion}
Die Ergebnisse aus dem Abschnitt \ref{sec:Auswertung} sind in folgenden Tabellen zusammengefasst.

\begin{table}[H]
  \centering
  \caption{Die Messwerte der Reflexion an einem Spiegel für verschiedene Winkel. Verwendet wurde der grüne Laser mit $\lambda = \SI{532}{\nano\meter}$.}
  \label{tab:MessungAufgabe1diskussion}
  \sisetup{table-format=2.1}
  \begin{tabular}{S S}
    \toprule
    {Einfallswinkel[\si{\degree}]} & {Ausfallswinkel[\si{\degree}]} \\
    \midrule
    69   &  69   \\
    42   &  43   \\
    30.5 & 30.5  \\
    55   & 56    \\
    72   & 72    \\
    20   & 20    \\
    35   & 35    \\
    52   & 52.5  \\
    \bottomrule
  \end{tabular}
\end{table}

\begin{table}[H]
  \centering
  \caption{Die Messwerte der Brechung an einer planparallelen Platte der Messung 1 für verschiedene Winkel. Verwendet wurde der grüne Laser mit $\lambda = \SI{532}{\nano\meter}$.}
  \label{tab:MessungAufgabe2diskussion}
  \sisetup{table-format=2.2}
  \begin{tabular}{S S}
    \toprule
    {Einfallswinkel[\si{\degree}]} & {Brechungswinkel[\si{\degree}]} \\
    \midrule
    69 & 39    \\
     0 &  0    \\
    10 &  7    \\
    30 & 20    \\
    40 & 25.5  \\
    75 & 40.5  \\
    55 & 33.5  \\
    38 & 24.25 \\
    \bottomrule
  \end{tabular}
\end{table}

\begin{equation}
  n = 1.476 \pm 0.010.
  \label{eqn:brechungsindexdiskussion}
\end{equation}

\begin{equation}
  p = 1.1 \pm 0.6 \si{\percent}.
  \label{eqn:abweichungdiskussion}
\end{equation}

\begin{equation}
  v=(2.031 \pm 0.013) \cdot 10^8
  \label{eqn:lichtgeschwdiskussion}
\end{equation}

\begin{table}[H]
  \centering
  \caption{Werte für den Strahlversatz, mit zwei Methoden berechnet.}
  \label{tab:strahlversatzdiskussion}
  \sisetup{table-format=3.3}
  \begin{tabular}
    {S S S S}
    \toprule
    {$\alpha [\si{\degree}]$} & {$s_1 [\si{\meter}]$} & {$s_2 [\si{\meter}]$} & {Abweichung $[\si{\percent}]$} \\
    \midrule
    45 & {$1.891 \cdot 10^{-3}$} & {$1.879 \cdot 10^{-3}$} & {$0.591$} \\
    60 & {$2.941 \cdot 10^{-3}$} & {$2.947 \cdot 10^{-3}$} & {$0.215$} \\
    20 & {$6.811 \cdot 10^{-4}$} & {$6.916 \cdot 10^{-4}$} & {$1.547$} \\
    51 & {$2.290 \cdot 10^{-3}$} & {$2.267 \cdot 10^{-3}$} & {$1.024$} \\
     3 & {$-5.115 \cdot 10^{-5}$} & {$9.893 \cdot 10^{-5}$} & {$293.4$} \\
    \bottomrule
  \end{tabular}
\end{table}

\begin{table}[H]
  \centering
  \caption{Die Messwerte der Brechung an einem Prisma für verschiedene Winkel. Verwendet wurde der grüne Laser mit $\lambda = \SI{532}{\nano\meter}$ und der rote Laser mit $\lambda = \SI{635}{\nano\meter}$.}
  \label{tab:MessungAufgabe4diskussion}
  \sisetup{table-format=2.1}
  \begin{tabular}{S S S}
    \toprule
    {Einfallswinkel[\si{\degree}]} & {Ausfallswinkel $L_{Gruen}$[\si{\degree}]} & {Ausfallswinkel $L_{Rot}$[\si{\degree}]} \\
    \midrule
    30  & 82   & 84   \\
    35  & 73.5 & 74.5 \\
    39  & 60   & 60.7 \\
    46  & 48   & 49.5 \\
    60  & 42   & 43   \\
    \bottomrule
  \end{tabular}
\end{table}

\begin{table}[H]
  \centering
  \caption{Die aus der Beugung berechneten Wellenlängen.}
  \label{tab:wellenlausw}
  \sisetup{table-format=3.3}
  \begin{tabular}
    {S S S S S S}
    \toprule
    {$\lambda_{600} [\si{\nano\meter}]$} & $p_{600} [\si{\percent}]$ & {$\lambda_{300} [\si{\nano\meter}]$} & $p_{300} [\si{\percent}]$ & {$\lambda_{100} [\si{\nano\meter}$} & $p_{100} [\si{\percent}]$ \\
    \midrule
      0.0   &   100     &     0.0    &    100    &     0.0   &  100     \\
    318.903 &   49.779  &   309.444  &    51.269 &   322.662 &  49.187  \\
            &           &   416.230  &    34.452 &   417.777 &  34.208  \\
            &           &            &           &   477.022 &  24.878  \\
            &           &            &           &   483.844 &  23.804  \\
            &           &            &           &   515.028 &  18.893  \\
            &           &            &           &   535.152 &  15.724  \\
            &           &            &           &   547.964 &  13.706  \\
    \bottomrule
  \end{tabular}
\end{table}
\noindent
Die in Aufgabenteil 1-4 berechneten Werte erscheinen plausibel und entsprechen im Rahmen der Messunsicherheiten
den Literaturwerten. Bei Aufgabe 5 jedoch ist die Abweichung von der angegebenen Wellenlänge des Lasers größer,
wobei die berechneten Werte von der Größenordnung mit dem idealen Wert übereinstimmen.
Es gibt mehrere Faktoren, die die Messung beeinflusst haben können.
\begin{itemize}
    \item \textit{Ablesegenauigkeit}
    \\\noindent
        Die Ablesegenauigkeit wurde stark durch die relativ große Skalierung der Winkelskala beschränkt.
        So kann je nach Versuch die Ablesegenauigkeit auf $\num{0.5}°$ bis $\num{1}°$ geschätzt werden.
        Die Ablesegenauigkeit wurde jedoch das Verrutschen der Vorlage zusätzlich verschlechtert, sodass
        auch systematische Fehler möglich sind. Dies kann unter anderem bei der Messung der Beugungsmaxima
        zu großen Ungenauigkeiten geführt haben, da die Halterung der Vorlage schlecht fixiert werden konnte.
        Damit könnte auch die relativ große Abweichung der berechneten von der angegebenen Wellenlänge
        erklären werden (vgl. Tabelle \ref{tab:wellenlausw}).
    \item \textit{Unzureichende Vorlagengröße}
    \\\noindent
        Bei der Messung der Beugungsmaxima trat zusätzlich das Problem auf, dass viele Maxima die Vorlage
        gar nicht erst getroffen haben, da diese zu klein war. Bei dem 300er- und 600er-Gitter führte das dazu,
        dass nur zwei von null verschiedene Maxima gemessen werden konnten.
    \item \textit{Ausrichtung des Lasers}
    \\\noindent
        Eine weitere mögliche Fehlerquelle ist die Ausrichtung des Lasers, auf die Mitte der Apparatur. Trifft
        der Laser leicht versetzt auf, führt dies zu einem systematischen Messfehler, der alle Ergebnisse
        beeinflusst. Da bei der Überprüfung des Reflexionsgesetzes keine systematischen Fehler ersichtlich sind,
        scheint der Laser die Mitte der Apparatur recht gut zu treffen.
\end{itemize}
