\section{Theorie}
\label{sec:theorie}
Bei Existenz eines Temperaturgefälles kommt es zu Wärmetransport hin zu
abnehmender Temperatur. Wärmetransport kann durch Konvektion, Wärmestrahlung oder
Wärmeleitung geschehen. Im folgenden Versuch wird nur die Wärmeleitung betrachtet.

Die Wärmemenge, die durch einen Stab der Länge $L$ und Querschnittsfläche $A$
mit einer Dichte $\rho$ und einer spezifischen Wärme $c$ transportiert wird,
lässt sich mit folgender Gleichung beschreiben:
\begin{equation}
  dQ = - \kappa A \frac{\partial T}{\partial x} dt.
  \label{eqn:wärmemenge}
\end{equation}
Hierbei bezeichnet $\kappa$ die materialabhängige Wärmeleitfähigkeit.
Daher folgt für die Wärmestromdichte
\begin{equation}
  j_w = - \kappa \frac{\partial T}{\partial x}.
  \label{eqn:wärmestromdichte}
\end{equation}
Unter Verwendung der Kontinuitätsgleichung ergibt sich die eindimensionale
Wärmeleitungsgleichung:
\begin{equation}
  \frac{\partial T}{\partial t} = \frac{\kappa}{\rho c} \frac{\partial^2 T}{\partial x^2}.
  \label{eqn:wärmeleitungsgleichung}
\end{equation}
Diese beschreibt die räumliche und zeitliche Entwicklung der Temperaturverteilung.
Die Temperaturverteilung $\sigma_T = \frac{\kappa}{\rho c}$ beschreibt die
Geschwindigkeit, mit der ein Temperaturgefälle ausgeglichen wird.
Durch eine periodische Erwärmung und Abkühlung eines sehr langen Stabes mit einer
festen Periode $T$ entsteht eine räumliche und zeitliche Temperaturwelle.
Diese lässt sich dann mit folgender Gleichung beschreiben:
\begin{equation}
  T(x,t) = T_{max} e^{-\sqrt{\frac{\omega \rho c}{2 \kappa}} x} \cos{\left(\omega t - \sqrt{\frac{\omega \rho c}{2 \kappa}} x \right)}.
  \label{eqn:temperaturwelle}
\end{equation}
