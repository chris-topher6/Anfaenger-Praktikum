\section{Diskussion}
\label{sec:Diskussion}
Die Abweichungen der Messungen an der Wheatstoneschen Brückenschaltung \ref{tab:aus:aA}
sowie der Kapazitätsmessbrücke \ref{tab:aus:bB} sind in derselben Größenordnung
wie die Unsicherheiten der Bauteile, die Messungen scheinen also recht genau zu sein.
Die Abweichungen der Induktivitätsmessbrücke \ref{tab:aus:cC} zeigen, dass
Induktivitäten aufgrund des Energieverlustes durch ihren recht hohen Innenwiderstand
nicht gut als Vergleichselement geeignet ist. Verglichen mit dem Ergebnis der
Maxwellbrücke der zweiten Messreihe \ref{tab:aus:dD} zeigt sich, dass diese
wesentlich genauer arbeitet als die Induktivitätsmessbrücke, was auf die bereits
genannten Gründe zurückzuführen ist. Das Ergebnis der ersten Messreihe der
Maxwellbrücke zeigt jedoch eine sehr hohe Abweichung von $\Delta = 207 \si{\percent}$.
Dies wurde vermutlich durch ein beschädigtes Bauteil ausgelöst. Die Verbindung zu
dem Oszilloskop oder der Tiefpass schienen einen Wackelkontakt zu haben, weshalb
diese zwischendurch gewechselt werden mussten. Da auch die Kabel und die anderen
Bauteile in einem nicht perfektem Zustand waren, kann auch hier die hohe Abweichung
entstanden sein.
Die in \ref{fig:ausw:e} dargestellte Theoriekurve weicht nur für hohe Werte von
den gemessenen Werten ab, der Abschnitt um die Sperrfrequenz zeigt nur recht
geringe Abweichungen. Dies deckt sich auch mit dem errechneten sehr geringen
Klirrfaktor.
