\section{Diskussion}
\label{sec:Diskussion}
Es folgt eine Zusammenfassung der Messergebnisse aus Kapitel \ref{sec:Auswertung}.
\begin{table}[H]
    \centering
        \caption{Zusammenfassung aller Parameter der Regressionen aus Kapitel \ref{sec:a1}.}
        \label{tab:diss1}
        \sisetup{table-format=1.3}
        \begin{tabular}{S[table-format=2.0] S[table-format=2.0] S @{${}\pm{}$} S S[table-format=2.2] @{${}\pm{}$} S[table-format=1.2]}
          \toprule
          {$\theta[°]$} & {$d_{Rohr} [\si{\milli\metre}]$} & \multicolumn{2}{c}{$a [\si{\hertz}/\text{rpm}]$} & \multicolumn{2}{c}{$b [\si{\kilo\hertz}]$} \\
          \midrule
          15 & 7  & -0.488 & 0.035 &  1.35 & 0.25 \\
          15 & 10 & -0.267 & 0.030 &  0.92 & 0.22 \\
          15 & 16 & -0.094 & 0.019 &  0.18 & 0.14 \\
          \cmidrule(lr){1-6}
          30 & 7  & 0.452 & 0.031  & -1.45 & 0.23 \\
          30 & 10 & 0.280 & 0.034  & -1.13 & 0.25 \\
          30 & 16 & 0.128 & 0.018  & -0.54 & 0.13 \\
          \cmidrule(lr){1-6}
          45 & 7  & -0.57  & 0.04   &  1.85 & 0.26 \\
          45 & 10 & -0.320 & 0.028  &  1.14 & 0.21 \\
          45 & 16 & -0.156 & 0.019  &  0.67 & 0.14 \\
          \bottomrule
       \end{tabular}
    \end{table}
\begin{table}[H]
    \centering
        \caption{Zusammenfassung der Parameter aus Kapitel \ref{sec:a2}.}
        \label{tab:diss2}
        \sisetup{table-format=3.1}
        \begin{tabular}{S[table-format=4.0]
            S[table-format=1.2] @{${}\pm{}$} S[table-format=1.2]
            S[table-format=1.3] @{${}\pm{}$} S[table-format=1.3]
            S[table-format=2.2] @{${}\pm{}$} S[table-format=2.2]
            S @{${}\pm{}$} S}
          \toprule
          & \multicolumn{4}{c}{$v_{stroem}$} & \multicolumn{4}{c}{$I$} \\
          \cmidrule(lr){2-5}\cmidrule(lr){6-9}
          {$P [\text{rpm}]$} &
          \multicolumn{2}{c}{$a [\si{\kilo\metre\per\square\second}]$} &
          \multicolumn{2}{c}{$b [\si{\metre\per\second}]$} &
          \multicolumn{2}{c}{$a [\si{\giga\square\volt\per\square\second}]$} &
          \multicolumn{2}{c}{$b [\si{\square\kilo\volt\per\square\second}]$} \\
          \midrule
          3870 &  0.00 & 0.00 & 0.000  & 0.000 &  7.19 &  2.42 & -89.5  &  37.9 \\
          6000 & -4.73 & 1.69 & 0.048  & 0.026 & 44.81 & 15.32 & -563.5 & 240.0 \\
          6000 & -0.03 & 0.75 & -0.032 & 0.012 & 18.79 & 3.99  &  187.9 &  39.9 \\
          \bottomrule
       \end{tabular}
    \end{table}

Die in Kapitel \ref{sec:a1} dargestellten Abbildungen \ref{fig:plot1}, \ref{fig:plot2} und \ref{fig:plot3} zeigen deutlich ein lineares Verhältnis
zwischen der Pumpleistung, und damit der Strömungsgeschwindigkeit, und der Frequenzverschiebung. Auffällig ist zudem, dass sich die Steigungen der
Regressionsgeraden bei gleichem Rohrdurchmesser in sehr hohem Maße gleichen. Dadurch, dass die Vorzeichen abhängig von der Strömungsrichtung sind,
können diese dabei vernachlässigt werden. Vermutlich bewirkt das Teilen der Frequenzverschiebung durch $\cos(\alpha)$ eine Unabhängigkeit der
Funktion von dem Prismawinkel $\theta$. Die vergleichsweise geringen Unsicherheiten der Parameter, welche durch \textit{nummeric python} \cite{numpy}
berechnt wurden, lassen auf keine großen Fehler bei der Aufnahme der Messdaten schließen.
\\\\\noindent
Die in Kapitel \ref{sec:a2} ausgewerteten Daten hingegen zeigen starke Auffälligkeiten. In Abbildung \ref{fig:plot4} ist zu erkennen, dass die
Messreihe für eine Pumpleistung von $45\%$ eine konstante Momentangeschwindigkeit von $v_{stroem}=0$ ergab. Da die Pumpe die Flüssigkeit jedoch
offensichtlich bewegte, sind diese Messdaten grundlegend falsch. Dies erklärt auch die Parameter der Regression, die allesamt null sind. Eine
mögliche Erklärung ist, dass das Messgerät nicht sensibel genug reagierte, um die Momentangeschwindigkeit der Flüssigkeit messen zu können.
Bei einer Pumpleistung von $70\%$ verschwand das Problem auch teilweise. Hier sind nur die ersten beiden Messwerte betroffen. Damit die
Messung trotzdem Aussagekraft besitzt, wurden in der Auswertung zwei Regressionsgeraden berechnet, wobei bei einer die offensichtlich fehlerhaften
Messdaten nicht mit in die Berechnung einflossen. Die großen Unsicherheiten der Parameter dieser bereinigten Regression deuten jedoch stark auf
weitere statistische Fehler bei der Messung hin.
\\\noindent
Die Messung der Intensität in Abhängigkeit von der Messtiefe zeigt deutlich geringere Unsicherheiten der Parameter. Zudem sind keine systematischen
Messfehler zu vermuten. Nur bei der Messung bei einer Pumpleistung von $70\%$ liegt ein Messwert deutlich abseits eines linearen Zusammenhangs, der
aus allen anderen Messdaten unmissverständlich hervorgeht. Deswegen wurden auch diesmal zwei Regressionen für diese Messreihe durchgeführt. Dabei
ist augenfällig zu erkennen, dass die Unsicherheit der Parameter durch den Ausschluss dieses Messwertes stark sinkt.
\\\noindent
Die Unsicherheiten, welche bei diesem Versuch generell recht gewichtig sind, lassen sich durch starke Fluktuationen während der Messung erklären,
die das Ablesen der zu messenden Werte stark erschwerten. Hinzu kommt, dass sich die Pumpleistung während der Messung unkontrolliert änderte. Dies
führt zu einer weiteren Verzerrung der Messergebnisse.
