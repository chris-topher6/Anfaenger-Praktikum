\section{Auswertung}
\label{sec:Auswertung}
\subsection{Messdaten}
\label{sec:Messdaten}
Die bei dem Versuch aufgenommenen Messdaten sind in den Tabellen \ref{tab_mess1} und \ref{tab:mess2} dargestellt. 
\begin{table}[H]
    \centering
        \caption{Die Frequnzverschiebung $\Delta\nu$ bei verschiedenen Pumpleistungen $P$ und Rohrdurchmessern $d_{Rohr}$}
        \label{tab:mess1}
        \sisetup{table-format=5.0}
        \begin{tabular}{S S S S S S S S S S}
          \toprule
          &
          \multicolumn{9}{c}{$\Delta\nu [\si{\hertz}]$}\\
          \cmidrule(lr){2-10}
          &
          \multicolumn{3}{c}{$d_{Rohr}=\SI{7}{\milli\metre} $}& %\quad\text{Rohr}$} & 
          \multicolumn{3}{c}{$d_{Rohr}=\SI{10}{\milli\metre}$}& %\quad\text{Rohr}$} &
          \multicolumn{3}{c}{$d_{Rohr}=\SI{16}{\milli\metre}$}\\ %\quad\text{Rohr}$} \\
          \cmidrule(lr){2-4}\cmidrule(lr){5-7}\cmidrule(lr){8-10}
          {$P [rpm]$} &
          {$15°$} & {$30°$} & {$45°$} &
          {$15°$} & {$30°$} & {$45°$} &
          {$15°$} & {$30°$} & {$45°$} \\
          \midrule
          6000 & -281 & 439 & -732  & -122 & 208 & -378 & -73 & 85  & -146 \\
          6500 & -317 & 482 & -879  & -134 & 200 & -415 & -73 & 98  & -159 \\
          7000 & -354 & 549 & -1038 & -171 & 281 & -525 & -73 & 110 & -183 \\
          7500 & -385 & 671 & -1172 & -183 & 305 & -635 & -98 & 122 & -220 \\
          8000 & -433 & 708 & -1227 & -195 & 391 & -647 & -98 & 159 & -269 \\
          8500 & -500 & 806 & -1440 & -244 & 415 & -745 & -110& 195 & -330 \\
          \bottomrule
        \end{tabular}
      \end{table}

\begin{table}[H]
  \centering
      \caption{Die Momentangeschwindigkeit $v$ und die Streuintensität $I$ bei Pumpleistungen von $45\%$ und $70\%$}
      \label{tab:mess2}
      \sisetup{table-format=3.0}
      \begin{tabular}{S S S S S}
        \toprule
        &
        \multicolumn{2}{c}{$P=45\%P_{max}$}&
        \multicolumn{2}{c}{$P=70\%P_{max}$}\\
        \cmidrule(lr){2-3}\cmidrule(lr){4-5}
        {$d_{mess} [\si{\micro\second}]$} &
        {$v [\si{\centi\metre\per\second}]$} & {$I[\si{\square\kilo\volt\per\second}]$} &
        {$v [\si{\centi\metre\per\second}]$} & {$I[\si{\square\kilo\volt\per\second}]$} \\
        \midrule
        12 &     0 & 11  & 0 & 5  \\
        13 &     0 & 33  & 0 & 6  \\
        14 & -28.7 & 73  & 0 & 8  \\
        15 & -35.0 & 108 & 0 & 15 \\
        16 & -36.6 & 125 & 0 & 26 \\
        17 & -31.8 & 111 & 0 & 24 \\
        18 & -31.8 & 117 & 0 & 17 \\
        19 & -31.8 & 470 & 0 & 75 \\
        \bottomrule
      \end{tabular}
    \end{table}

\subsection{Frequnzverschiebung bei verschiedenen Rohrdurchmessern}
\label{sec:a1}
In den Abbildungen \ref{fig:plot1}, \ref{fig:plot2} und \ref{fig:plot3} sind die Messwerte aus Tabelle \ref{tab:mess1} graphisch ausgewertet. 
Dabei sind auf die x-Achsen die Pumpleistungen $P$ und auf die y-Achsen $\Delta\nu/\cos(\alpha)$ aufgetragen, wobei $\alpha$ der jeweilige
Dopplerwinkel aus \ref{sec:vorbereitung} ist. In jeder Abbildung ist dann für jeden Rohrdurchmesser eine lineare Regression eingetragen.
Diese wurde von \textit{nummeric python} \cite{numpy} berechnet und haben die Form 
\begin{equation}
    y=mx+b  . 
    \label{eqn:gleichung}
\end{equation}
Die Parameter sind in Tabelle \ref{tab:params1} aufgelistet.


%Die Tabelle sieht total kacke aus...
%nochmal überarbeiten
\begin{table}[H]
    \centering
        \caption{Die Parameter der linearen Regressionen in den Abbildungen \ref{fig:plot1}-\ref{fig:plot3}}
        \label{tab:mess1}
        \sisetup{table-format=2.3}
        \begin{tabular}{S S @{${}\pm{}$} S S @{${}\pm{}$} S S @{${}\pm{}$} S S @{${}\pm{}$} S S @{${}\pm{}$} S S @{${}\pm{}$} S}
          \toprule
          &
          \multicolumn{4}{c}{$d_{Rohr}=\SI{7}{\milli\metre}$}&
          \multicolumn{4}{c}{$d_{Rohr}=\SI{10}{\milli\metre}$}&
          \multicolumn{4}{c}{$d_{Rohr}=\SI{16}{\milli\metre}$}\\
          \cmidrule(lr){2-5}\cmidrule(lr){6-9}\cmidrule(lr){10-13}
          {$\theta [°]$} &
          {$a[\si{\hertz\per\second}]$} & {$b[\si{\kilo\hertz}]$} &  %Einheiten unbedingt in rpm ändern
          {$a[\si{\hertz\per\second}]$} & {$b[\si{\kilo\hertz}]$} &
          {$a[\si{\hertz\per\second}]$} & {$b[\si{\kilo\hertz}]$} \\
          \midrule
            15 & -0.488 & 0.035 &   1.35 & 0.25  & -0.267 & 0.030  &  0.92 & 0.22 & -0.094 & 0.019 &  0.18 & 0.14 \\
            30 &  0.452 & 0.031  & -1.45 & 0.23  &  0.280 & 0.034  & -1.13 & 0.25 &  0.128 & 0.018 & -0.54 & 0.13 \\
            45 &  -0.57 & 0.04   &  1.85 & 0.26  & -0.320 & 0.028  &  1.14 & 0.21 & -0.156 & 0.019 &  0.67 & 0.14 \\
          \bottomrule
        \end{tabular}
      \end{table}
\subsection{Strömungsprofil}
\label{sec:a2}