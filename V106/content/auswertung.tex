\section{Auswertung}
  \subsection{Die Schwingungsdauern T1 und T2}
      Die Schwingungsdauern der frei schwingenden Pendel werden in folgender Tabelle wiedergegeben:
        \begin{table}
          \centering
            \caption{freie Schwingungsdauern T1 und T2.}
            \label{tab:aufgabe1}
            \sisetup{table-format=1.4}
            \begin{tabular}{S S}
              \toprule
              {$T_{1}$} & {$T_{2}$} \\
              \midrule
              1.9400 &     1.9450 \\
              \bottomrule
            \end{tabular}
          \end{table}
       Die dem zugrundeliegenden Messdaten beziehen sich auf 5 Schwingungen. Es wurden zehn Messungen für jeweils Pendel 1
       und Pedel 2 durchgeführt, wobei die eingestellte Pendellänge von Pendel 1 $0.993m$ und von Pendel2 $0.995m$ beträgt.
       Gemessen wurde die Pendellänge vom Gewichtsmittelpunkt bis zu der Auflagenadel des Pendels. Die  Schwingungsdauern
       wurden berechnet, indem der Mittelwert der 10 Messungen durch 5 geteilt wurde.
       Problematisch bei diesen Messungen ist, dass die Federhöhe in den Anweisungen nicht genauer spezifiziert ist.
       Solcherlei Ungenauigkeiten in der Anweisung werden im folgenden an den entsprechenden Punkten genannt und in der
       Diskussion abschließend behandelt. Bei obigen Messungen hing die Feder $0.288m$ unter der Pendelaufhängung.
  \subsection{Schwingungsdauern für gleich- und gegensinnige Schwingungen}
    \begin{table}
      \centering
        \caption{gegensinnige Schwingungsdauer $T_{-}$ und gleichsinnige Schwingungsdauer $T_{+}$.}
        \label{tab:aufgabe23}
        \sisetup{table-format=1.4}
        \begin{tabular}{S S}
          \toprule
          {$T_{+}$} & {$T_{-}$ Versuch 1} & {$T_{-}$ Versuch 2}\\
          \midrule
          
\label{sec:Auswertung}
