\section{Theorie}
\label{sec:Theorie}
\subsection{Der harmonische Oszillator}
Ein harmonischer Oszillatorist ein physikalisches System, welches eine Schwingung ausführt. Das einfachste Beispiel
hierfür ist eine Feder, an welcher eine Masse  $m$ um die Ruhelage $x_0$ oszilliert.
%Abbildung einfügen
Ein harmonischer Oszillator kann dabei über die Differentailgleichung
\begin{equation}
    \ddot{x}+\omega^2\cdot x=0
\end{equation}
definiert werden. Im mechanischen Fall ist dabei $x$ der Ort, $\ddot{x}$ die zweite Ableitung des Ortes nach der Zeit,
also die Beschleunigung, und $\omega$ eine Konstante, welche weitere Informationen über das System, wie zum Beispiel die 
Masse $m$, ergänzt. Zudem wird sich noch zeigen, dass $\omega$ die Frequenz der Schwingung beschreibt.
\\
Die Differentailgleichung kann bei der Feder aus dem zweiten Newton'schen Axiom
\begin{equation}
    F=m\cdot\dddot{x} \label{eqn:newton2}
\end{equation} 
hergeleitet werden. Dafür wird das zweite Newton'sche Axiom der rücktreibenden Kraft der Feder gleichgesetzt. Die 
Gleichheit der beiden Kräfte folgt aus dem Reaktionsprinzip. Die rücktreibende Kraft wird bei dem obrigen Beispiel
durch das Hook'sche Gesetz beschrieben
\begin{equation}
    F=-k\cdot x \label{eqn:hook}. %Vorzeichen???
\end{equation}
Dabei ist $k$ die Federkonstante der Feder.
\\
Durch das Gleichsetzen erhält man:
\begin{equation}
    m\cdot\dddot{x}=k\cdot x \Leftrightarrow \ddot{x}+\frac{k}{m}\cdot x=0 \label{eqn:dgl}
\end{equation}
Gesucht ist nun die Funktion $x(t)$, die die Differentailgleichung löst. Man wählt nun den Ansatz
\begin{align}
    x(t)&=A\cdot cos(\omega\cdot t)+B\cdot sin(\omega\cdot t) \\
    \ddot{x}(t)&=-\omega^2\cdot x(t)
    \label{eqn:ansatz}
\end{align}
$A$ und $B$ sind dabi Konstanten, die durch die Anfangsbedingung bestimmt werden. Einsetzen des Ansatzes \eqref{eqn:ansatz}
in die Differentailgleichung \eqref{eqn:dgl} ergibt dann:
\begin{equation}
    A\cdot cos(\omega\cdot t)+B\cdot sin(\omega\cdot t)(\frac{k}{m}-\omega^2)=0
\end{equation}
Da der erste Fraktor der Gleichung nicht für alle Zeiten $t=0$ sein kann, muss der zweite Faktor null ergeben
\begin{equation}
    \frac{k}{m}-\omega^2=0 \Leftrightarrow \frac{k}{m}=\omega^2 .
\end{equation}
