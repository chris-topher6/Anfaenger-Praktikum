\section{Durchführung}
\label{sec:Durchführung}
\subsection{Versuchsaufbau}
Für den Versuch werden zwei identische Stabpendel nebeneinander aufgebaut. Eines der Pendel besteht dabei aus einer langen Metallstange, an welcher man eine
zylinderförmige Masse $m=(\SI{1}{\kilo\gram})$ verschieben kann. Dadurch kann das Pendel unterschiedliche Längen annehmen. In der Metallstange sind zusätzlich
noch drei Löcher eingearbeitet, welche als Halterung für eine Feder dienen. Durch Anbringen der Feder in die jeweils gleichen Löcher der Pendel können diese
gekoppelt werden.
\\
Um die Pendel möglichst gut mit einem harmonischen Oszillator annähern zu können, wird bei dem Pendel durch eine spezielle Aufhängung die Reibung minimiert. 
Die sogenannte Spitzenlagerung besteht aus zwei Spitzen, die sich auf einer Metallstange in einer Kerbe befinden. 
\\
Um die Periodendauer zu messen wird eine Stoppuhr verwendet und die Messung der Pendellänge wird mit einem Maßband durchgeführt.
\subsection{Versuchsdurchführung}
\begin{itemize}
    \item Es wird für jedes Pendel zehn Mal die Periodendauer $T$ gemessen, indem die Zeit zwischen fünf Schwingungen gestoppt wird. Dabei sind die Pendel 
        \textit{nicht} mit einer Feder gekoppelt. Die Pendellänge der beiden Pendel soll dabei übereinstimmen. Um dies zu überprüfen werden die Periodendauern 
        $T_1$ und $T_2$ mit einander verglichen. Stimmen die Werte nicht im Rahmen der Messungenauigkeiten überein, müssen die Pendellängen angepasst und die 
        Messungen wiederholt werden. 
    \item Nun werden die Pendel mittels einer Feder gekoppelt. Gemessen wird nun sowohl die Schwingungsdauer $T_-$, der gegensinnigen Schwingung, als auch die 
        Schwingungsdauer $T_+$ der gleichsinnigen Schwingung. Die Messung erfolgt wieder zehn Mal, wobei wieder fünf Schwingungen gemessen werden. 
    \item Nun werden die Periodendauer $T$ und die Schwebungsdauer $T_S$ für eine gekoppelte Schwingung gemessen. Die Messungen erfolgen wieder zehn Mal für 
        jeweils fünf Schwingungen.
    %Eigenfrequenzen??????
    \item Die Pendellängen werden nun verändert, und es werden noch einmal alle Messungen für die neue Pendellänge durchgeführt.
\end{itemize}
