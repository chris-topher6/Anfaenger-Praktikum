\section{Auswertung}
  \subsection{Abmessungen und elektrischer Widerstand der Probe}
    Die untersuchte Silberprobe ist $0.022 \si{\milli\meter}$ dick und $14.844 \si{\milli\meter}$ lang sowie
    $25.164\si{\milli\meter}$ breit. Der gemessene elektrische Widerstand beträgt $0.6 \si{\ohm}$.
  \subsection{Untersuchung des Hall-Effektes}
    \begin{table}
      \centering
        \caption{Messung der Hall-Spannung mit konstant gehaltenem Querstrom mit $\SI{5}{\ampere}$.}
        \label{tab:hallspannung1}
        \sisetup{table-format=1.4}
        \begin{tabular}{S S S }
          \toprule
          {$I_{Feld} /\si{\ampere}$} & {$U_{Hall} /\si{\milli\volt}$} & {$U_{Hall} / \si{\milli\volt}$, umgepoltes Magnetfeld} \\
          \midrule
          0   & 0.0871 & 0.0031 \\
          0.5 & 0.0885 & 0.0857 \\
          1.0 & 0.0897 & 0.0845 \\
          1.5 & 0.0914 & 0.0829 \\
          2.0 & 0.0926 & 0.0812 \\
          2.5 & 0.0942 & 0.0798 \\
          3.0 & 0.0957 & 0.0785 \\
          3.5 & 0.0971 & 0.0779 \\
          4.0 & 0.0989 & 0.0759 \\
          4.5 & 0.0992 & 0.0748 \\
          5.0 & 0.1002 & 0.0738 \\
          \bottomrule
        \end{tabular}
      \end{table}
      \begin{table}
        \centering
          \caption{Messung der Hall-Spannung mit konstant gehaltenem Magnetfeld mit $\SI{5}{\ampere}$.}
          \label{tab:hallspannung1}
          \sisetup{table-format=1.4}
          \begin{tabular}{S S S }
            \toprule
            {$I_{q} /\si{\ampere}$} & {$U_{Hall} /\si{\milli\volt}$} & {$U_{Hall} / \si{\milli\volt}$, umgepoltes Magnetfeld} \\
            \midrule
            0   & 0.0015 & 0.0016 \\
            0.5 & 0.0115 & 0.0088 \\
            1.0 & 0.0213 & 0.0158 \\
            1.5 & 0.0311 & 0.0231 \\
            2.0 & 0.0409 & 0.0300 \\
            2.5 & 0.0507 & 0.0373 \\
            3.0 & 0.0605 & 0.0443 \\
            3.5 & 0.0703 & 0.0513 \\
            4.0 & 0.0802 & 0.0585 \\
            4.5 & 0.0900 & 0.0657 \\
            5.0 & 0.0999 & 0.0727 \\
            \bottomrule
          \end{tabular}
        \end{table}
  \subsection{Berechnung der Leitfähigkeitsparameter}

\label{sec:Auswertung}
