\section{Auswertung}
\label{sec:Auswertung}
\subsection{Die mittlere freie Weglänge}
Die mittlere freie Weglänge $\bar{w}$ wurde nach \eqref{eqn:freieweglaengetheorie} aus den Temperaturen berechnet.
Die dafür benötigte Berechnung des Sättigungsdampfdruckes wurde nach \eqref{eqn:sättigungsdampfdrtheorie} durchgeführt.
In folgender Tabelle sind die Weglängen für die verschiedenen Temperaturen aufgeführt.

\begin{table}[H]
  \centering
    \caption{Die mittlere freie Weglänge für verschiedene Temperaturen.}
    \label{tab:freieweglaengeausw}
    \sisetup{table-format=3.2}
      \begin{tabular}{S S[table-format=2.3] S[table-format=2.3] S[table-format=2.3]}
        \toprule
        {Temperatur [$\si{\kelvin}$]} & {$p_\text{Sätt}$}  & {$\bar{w}$ [$\si{\milli\bar}$]} & {$\frac{a}{\bar{w}}$} \\
        \midrule
         297.15  &    \num{4.908e-03}      &          0.591  &  \num{1.692e-2} \\
         417.05  &              3.802      & \num{7.627e-4}  &          13.111 \\
         447.25  &             11.576      & \num{2.505e-4}  &          39.916 \\
         471.15  &             25.248      & \num{1.149e-4}  &          87.064 \\
        \bottomrule
      \end{tabular}
    \end{table}

\subsection{Die differentielle Energieverteilung der Elektronen}
Um die differentielle Energieverteilung der Elektronen zu erhalten, muss zuerst die Skalierung ermittelt werden.
Dafür wurde die Spannung bei $\SI{0}{\volt}, \SI{1}{\volt}, ..., \SI{10}{\volt}$ auf der Abzisse eingezeichnet.
Aus der Breite dieser $\SI{1}{\volt}$-Abstände wird der Mittelwert gebildet, um eine Relation zwischen
$\si{\milli\meter}$ und $\si{\volt}$ zu erhalten. Dies wurde für die beiden Messreihen durchgeführt und ist in
folgender Tabelle zusammengefasst.

\begin{table}[H]
  \centering
  \caption{Die Skalierung der beiden Messreihen.}
  \label{tab:skalierungausw}
  \sisetup{table-format=2.2}
    \begin{tabular}{S S S}
      \toprule
      {Abschnitt [$\si{\volt}$]} & {Breite Messreihe 1 [$\si{\milli\meter}$]} & {Breite Messreihe 2 [$\si{\milli\meter}$]} \\
      \midrule
      {(0-1)}  &  23   & 25.5  \\
      {(1-2)}  &  22   & 22  \\
      {(2-3)}  &  22.5 & 23  \\
      {(3-4)}  &  23.5 & 21.5  \\
      {(4-5)}  &  23   & 24  \\
      {(5-6)}  &  23   & 24  \\
      {(6-7)}  &  23.5 & 24  \\
      {(7-8)}  &  24   & 24  \\
      {(8-9)}  &  23.5 & 23  \\
      {(9-10)} &  25   & 26  \\
      {$\rightarrow$} & & \\
      \bottomrule
    \end{tabular}
  \end{table}
