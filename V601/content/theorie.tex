\section{Theorie}
\label{sec:Theorie}
\subsection{Wechselwirkungen zwischen Atomen und Elektronen}
\label{sec:wechsel}
Der Franck-Hertz-Versuch ist ein Elektronenstoßexperiment. Es treffen also Elektronen auf Hg-Atome und führen dabei elastische und 
inelastische Stöße mit diesen aus. Die Art des Stoßes ist dabei abhängig von der kinetischen Energie $E$ des Elektrons. Bei dem elastische
Stoß wird das Elektron dabei näherungsweise an dem Hg-Atom gestreut, bei dem unelastischen Stoß regt das Elektron das Hg-Atom an. \\
\subsection{Die Gegenfeldmethode}
\label{sec:gegen}
\subsection{Störeinflüsse}
\label{sec:stör}