f\section{Theorie}
\label{sec:Theorie}
%ich brauche vor allem Gleichung 8, 9 und 10 aus der Anleitung. Und am besten wäre eine Herleitung von den Flächenträgheitsmomenten.
%Nicolai hatte das ja durchgerechnet und das ging ja auch ziemlich fix.
\section{Problemstellung}
Eine Kraft, die an eine Oberfläche angreift, erzeugt pro Flächeneinheit eine Spannung.
Diese kann senkrecht oder aber tangential zu der Oberfläche wirken und wird dann
Normalspannung oder Tangential- beziehungsweise Schubspannung genannt.
Der Zusammenhang zwischen der Normalspannung $\sigma$ und
der erzeugten Längenänderung $\Delta L/L$ eines Körpers wird beschrieben durch das Hookesche Gesetz.
Es ergibt sich zu
\begin{equation}
  \sigma = E \frac{\Delta L}{L}.
  \label{eqn:captainhook}
\end{equation}
Hierbei meint $E$ das Elastizitätsmodul, einen Proportionalitätsfaktor, der eine
Materialkonstante des jeweiligen Körpers ist und der experimentell mithilfe der
Biegung eines Körpers bestimmt werden kann.
Hierfür beschreibt man die Biegung mit einer Funktion $D(x)$. Für diese gilt
\begin{equation}
  D(x) = D_G(x) + D_0(x).
  \label{eqn:DmitNull}
\end{equation}
Hierbei bezeichnet $D_G(x)$ die Auslenkung bei einem angehängten Gewicht an den Körper
und $D_0(x)$ die Auslenkung des Körpers in der Ruhelage ohne ein angehängtes Gewicht.
Um die Funktion zu bestimmen, können zwei Verfahren verwendet werden. Auf diese
wird im Folgenden näher eingegangen.
