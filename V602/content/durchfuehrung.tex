\section{Durchführung}
\label{sec:Durchführung}
\subsection{Vorbereitungsaufgaben}
\label{sec:vorbereitung}
Die in der Vorbereitungsaufgabe gefragten Literaturwerte bezüglich der Ordnungszahl $Z$, der Energie der K-Kante $E_K$, des Braggwinkels
$\theta_K$ und der Abschirmkonstante $\sigma$ sind im Folgenden tabellarisch dargestellt.
\begin{table}[H]
    \centering
        \caption{Die gefragten Literaturwerte zu verschiedenen Elementen.\cite{AP05}}
        \label{tab:diss11}
        \sisetup{table-format=1.2}
        \begin{tabular}{S S[table-format=2.0] S[table-format=2.2] S[table-format=2.1] S}
          \toprule
          {Element} & {$Z$} & {$E_K [\si{\kilo\electronvolt}]$} & {$\theta [\si{\degree}]$} & {$\sigma$}\\
          \midrule
            {Zn} & 30 & 9.65  & 18.6 & 3.56 \\
            {Ge} & 32 & 11.10 & 18.2 & 3.68 \\
            {Br} & 35 & 14.47 & 15.0 & 3.85 \\
            {Rb} & 37 & 15.20 & 13.3 & 3.95 \\
            {Sr} & 38 & 16.10 & 12.6 & 4.01 \\
            {Zk} & 40 & 17.99 & 11.3 & 4.11 \\
          \bottomrule
        \end{tabular}
      \end{table}
\subsection{Der Versuchsaufbau}
\label{sec:Versuchsaufbau}
Der Versuchsaufbau ist in folgender Grafik zu erkennen.
\begin{figure}[H]
  \centering
  \includegraphics[scale=0.4]{"content/aufbau.png"}
  \caption{Der Versuchsaufbau. \cite{AP01}}
  \label{fig:aufbaudurchführung}
\end{figure}
\noindent
Der Versuchsaufbau besteht aus einer Kupfer-Röntgenröhre, welche auf einen LiF-Kristall
ausgerichtet ist. Der Kristall lässt sich auf verschiedene Winkel einstellen,
wodurch die Intensität der Röntgenstrahlung sich für verschiedene Glanzwinkel
messen lässt. Zur Intensitätsmessung wird ein Geiger-Müller Zählrohr verwendet.
Die Messung sowie die Winkeleinstellung lassen sich über die Bedienelemente des
Gerätes steuern. Die Röntgenröhre wird auf eine Beschleunigungsspannung von
$U_B = 35 \si{\kilo\volt}$ und einen Emissionstrom von $I = 1 \si{\milli\ampere}$
eingestellt. Um dann die Absorptionsmessung durchzuführen, werden verschiedene
Blenden mit verschiedenen Absorbern vor dem Geiger-Müller Zählrohr angebracht.
Dabei muss die Schlitzblende in Drehrichtung senkrecht ausgerichtet sein.

\subsection{Überprüfung der Bragg Bedingung}
Zur Überprüfung der Bragg Bedingung wird der LiF-Kristall auf einen festen Winkel
von $\Theta = 14 \si{\degree}$ eingestellt. Dann wird mit dem Geiger-Müller Zählrohr
in einem Winkelbereich von $\alpha_{GM} = 26 \si{\degree}$ bis $\alpha_{GM} = 30 \si{\degree}$
bei einem Zuwachs von $\Delta \alpha = 0.1 \si{\degree}$ die Intensität gemessen.
Die Integrationszeit pro Winkel beträgt $\Delta t = 5 \si{\second}$.

\subsection{Das Emissionspektrum der Cu-Röntgenröhre}
Um das Emissionsspektrum der Kupferröhre aufzunehmen, wird im Programm der \textit{2:1 Koppelmodus}
ausgewählt. Dann wird das Röntgenspektrum bei einer Beugungsordnung $n=1$ im Bereich
$4 \si{\degree} \leq \Theta \leq 26 \si{\degree}$ aufgenommen. Die Integrationszeit
beträgt pro Winkel $\Delta t = 5 \si{\second}$.

\subsection{Das Absorptionsspektrum}
Für die Messung des Absorptionspektrums wird der Zinkabsorber vor dem Geiger-Müller
Zählrohr positioniert. Das Absorptionsspektrum wird in $0.1 \si{\degree}$ ausgemessen.
Die Integrationszeit beträgt hierbei $\Delta t = 20 \si{\second}$.
Die Messungen werden für vier weitere Absorber mit Kernladungszahlen im Bereich
$30 \leq Z \leq 50$ wiederholt.
