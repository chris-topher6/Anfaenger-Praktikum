\section{Diskussion}
\label{sec:Diskussion}
Verglichen mit dem Literaturwert $\frac{h}{e_0}_{lit} = 4.136 \cdot 10^{-15}
\si{\kilo\gram\meter\squared\per\second\squared\ampere}$ \cite{AP03}
ergibt sich eine Abweichung von $\SI{150}{\percent}$. Dies ist sehr hoch,die Werte liegen
aber in derselben Größenordnung ($10^{-15}$). Die hohen Abweichungen sowie die
negativen Werte für die Grenzspannung, das Verhältnis $\frac{h}{e_0}$ und die
Austrittsarbeit lassen auf einen systematischen Fehler schließen, der jede Messung
leicht verfälscht hat. Dies hat vermutlich einen besonders hohen Einfluss auf den
Fit an Tabelle \ref{tab:werteb}, aufgrund der geringen Anzahl der Werte, an die gefittet wird.
Zusätzlich fiel bereits während der Messung auf, dass Faktoren wie das Öffnen und
Schließen des Vorhangs und Bewegungen einer anderen experimentierenden Gruppe einen
hohen Einfluss auf die gemessenen Werte hatten. Außerdem ist der Versuchsaufbau
sehr anfällig gegenüber Stößen an den Tisch, was ebenfalls die Messwerte verfälscht
haben könnte.
