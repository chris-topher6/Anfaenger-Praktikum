\section{Durchführung}
\label{sec:Durchführung}

\subsection{Versuchsaufbau}
\label{sec:Aufbau}
Wie bereits in Kapitel \ref{sec:nachweis} beschrieben, ist in dem Versuchsaufbau eine evakuierte Photozelle (vgl. Abb. \ref{fig:Photozelle}) mit einem
Picoamperemeter und einer variablen Spannungsquelle verbunden (vgl. Abb. \ref{fig:Schaltung}). Da das Licht, mit welchem die Kathode
bestrahlt wird, aber eine feste Frequenz $\nu$ besitzen soll, wird auch eine optische Konstruktion benötigt.
\begin{figure}
    \centering
    \includegraphics[scale=0.3]{pictures/OptischerTeil.png}
    \caption{Optische Konstruktion zur Gewinnung monochromatischen Lichts. \cite{AP01}}
    \label{fig:optisch}
\end{figure}
Wie in Abbildung \ref{fig:optisch} zu sehen, wird von einer
Spektrallampe ausgesendetes polychromatisches Licht durch Linsen und einen Spalt fokussiert und dann mit einem Prisma in seine monochromatischen
Spektrallinien zerlegt. Die Photozelle kann jetzt so auf einem Schwenkarm bewegt werden, dass immer nur eine Spektrallinie auf die Kathode
fällt. Somit können für mehrere Lichtfrequenzen Messungen durchgeführt werden.

\subsection{Ausmessung des Photostroms}
\label{sec:ausmessen}
Vor Beginn der Messung muss der Dunkelstrom gemessen werden, welcher das Picoamperemeter auch ohne Anschalten der Lampe anzeigt, da dieser
die Messung sonst systematisch verfälschen würde. Erst dann wird die Spektrallampe angeschaltet.
\\
\\
\noindent
Nach Ausrichten einer Spektrallinie auf die Photozelle wird die Spannung von $\SI{2}{\volt}$ bis
$\SI{-2}{\volt}$ in $\SI{0.2}{\volt}$-Schritten variiert. Dabei werden die gemessenen Spannungen abgelesen und notiert. Diese Messung
erfolgt für die grüne ($\lambda=\SI{546}{\nano\metre}$), die blaugrüne ($\lambda=\SI{492}{\nano\metre}$) und die eine violette
($\lambda=\SI{405}{\nano\metre}$, $\lambda=\SI{435}{\nano\metre}$) Spektrallinie.
\\
\\
\noindent
Für die gelbe Spektrallinie ($\lambda=\SI{577}{\nano\metre}$) wird die Spannung in einem Bereich von $\SI{20}{\volt}$ bis $\SI{-2}{\volt}$
variiert, wobei in dem Bereich von $\SI{20}{\volt}$ bis $\SI{2}{\volt}$ eine Schrittweite von $\SI{1}{\volt}$ gewählt wird.
