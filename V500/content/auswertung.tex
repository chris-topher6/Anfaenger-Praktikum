\section{Auswertung}
\label{sec:Auswertung}
Die gemessenen Werte sind in folgenden Tabellen dargestellt.

Es wird ein Dunkelstrom von $I_{Dunkel} = 0.03 \si{\nano\ampere}$ gemessen.
Dieser wird vor weiteren Rechnungen von allen gemessenen Strömen abgezogen.
\noindent
Für die grüne Spektrallinie mit einer Wellenlänge von $\lambda = 546 \si{\nano\meter}$
ist in folgender Abbildung die Wurzel des gemessenen Photostroms gegen die Bremsspannung
aufgetragen.
\begin{figure}[H]
  \centering
  \includegraphics[scale=0.8]{build/plotgrün.pdf}
  \label{fig:plotgrün}
  \caption{Die Messwerte der grünen Spektrallinie sowie ein linearer Fit.}
\end{figure}
\noindent
An die Messwerte ist eine lineare Funktion der Form $mx +b$ gefittet. Es ergeben
sich die Parameter
\begin{align*}
  m = & 0.543 \pm 0.029 \\
  b = & 0.456 \pm 0.029. \\
\end{align*}
Die Grenzspannung $U_g$ ergibt sich hier zu $ U_g = -0.84 \pm 0.07$ \si{\nano\ampere}. \\
\noindent
Für die blaugrüne Spektrallinie mit einer Wellenlänge von $\lambda = 492 \si{\nano\meter}$
ist in folgender Abbildung die Wurzel des gemessenen Photostroms gegen die Bremsspannung
aufgetragen.
\begin{figure}[H]
  \centering
  \includegraphics[scale=0.8]{build/plotblaugrün.pdf}
  \label{fig:plotblaugrün}
  \caption{Die Messwerte der blaugrünen Spektrallinie sowie ein linearer Fit.}
\end{figure}
\noindent
An die Messwerte ist eine lineare Funktion der Form $mx +b$ gefittet. Es ergeben
sich die Parameter
\begin{align*}
  m = & 0.141 \pm 0.005 \\
  b = & 0.137 \pm 0.005.\\
\end{align*}
Die Grenzspannung $U_g$ ergibt sich hier zu $ U_g = -0.97 \pm 0.05$ \si{\nano\ampere}. \\
\noindent
Für die blaue Spektrallinie mit einer Wellenlänge von $\lambda = 435 \si{\nano\meter}$ \cite{AP02}
ist in folgender Abbildung die Wurzel des gemessenen Photostroms gegen die Bremsspannung
aufgetragen.
\begin{figure}[H]
  \centering
  \includegraphics[scale=0.8]{build/plotblau.pdf}
  \label{fig:plotblau}
  \caption{Die Messwerte der blauen Spektrallinie sowie ein linearer Fit.}
\end{figure}
\noindent
An die Messwerte ist eine lineare Funktion der Form $mx +b$ gefittet. Es ergeben
sich die Parameter
\begin{align*}
  m = & 0.284 \pm 0.007 \\
  b = & 0.346 \pm 0.007.\\
\end{align*}
Die Grenzspannung $U_g$ ergibt sich hier zu $ U_g = -1.22 \pm 0.04$ \si{\nano\ampere}. \\
\noindent
Für die gelbe Spektrallinie mit einer Wellenlänge von $\lambda = 577 \si{\nano\meter}$
ist in folgender Abbildung die Wurzel des gemessenen Photostroms gegen die Bremsspannung
aufgetragen.
\begin{figure}[H]
  \centering
  \includegraphics[scale=0.8]{build/plotgelb.pdf}
  \label{fig:plotgelb}
  \caption{Die Messwerte der gelben Spektrallinie sowie ein linearer Fit.}
\end{figure}
\noindent
An den Teil der Messwerte, welcher näherungsweise linear verläuft, ist eine lineare Funktion der Form $mx +b$ gefittet. Es ergeben sich die Parameter
\begin{align*}
  m = & 0.0138 \pm 0.0008 \\
  b = & 1.266 \pm 0.011.\\
\end{align*}
Die Grenzspannung $U_g$ ergibt sich hier zu $ U_g = -92 \pm 5$ \si{\nano\ampere}. \\
