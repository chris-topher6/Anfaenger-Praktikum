\section{Theorie}
\label{sec:Theorie}
Auf Grund verschiedener experimenteller Befunde kann Licht nur durch die Quantenelektrodynamik widerspruchsfrei beschrieben werden.
Für eine große Anzahl an Photonen kann Licht dabei als Welle genähert werden, wie es zum Beispiel in der Beugungsoptik geschieht.
Wechselwirkt das Licht jedoch mit Materie, wie es bei dem Photoeffekt der Fall ist, können die auftretenden Phänomene mit dem
Korpuskelmodell erklärt werden, in welcher die Quantelung des Lichts in einzelne Photonen angenommen wird. Im Folgenden soll
das Licht nach dem Korpuskelmodell behandelt werden.

\subsection{Die Phänomene und Erklärung des Photoeffektes}
\label{sec:phänomene}
Wird eine Metalloberfläche mit Photonen der Energie
\begin{equation}
    E_{P}=h\nu
    \label{eqn:photon}
\end{equation}
bestrahlt, können durch diese Elektronen aus dem Metall herrausgelöst werden. Dabei überträgt das Photon seine Energie dem im Metall
befindlichen Elektron. Zum Verlassen des Metalls muss die spezifische
Austrittsarbeit $\Phi_K$ geleistet werden. Die restliche Energie wird in kinetische Energie $E_{kin}\geq 0$ des Elektrons umgesetzt. Die
Energiebilanz für ruhende Elektronen (vgl. Abschnitt \ref{sec:nachweis} für bewegte Elektronen) lautet demnach
\begin{equation}
    h\nu=\Phi_K+E_{kin}\;.
    %h\nu+E_{kin}=\Phi_K+E'_{kin}\;.
    \label{eqn:bilanz}
\end{equation}
Die Energie $E_{kin}$ der Elektronen nach dem Stoß ist also proportional zur Frequenz $\nu$ des Lichts.
Offensichtlich kann der Photoeffekt nur auftreten, wenn
\begin{equation*}
    h\nu<\Phi_K
\end{equation*}
gilt. Es existiert also eine minimale Grenzfrequenz unterhalb derer keine Elektronen mehr aus dem Metall gelöst werden können.
Des Weiteren ist zu beobachten, dass die Anzahl der ausgelösten Elektronen proportional zur Intensität des Lichtes ist.

\subsection{Experimenteller Nachweis des Photoeffektes}
\label{sec:nachweis}
Für den Nachweis des Photoeffektes müssen die ausgelösten Elektronen gemessen werden. Dies geschieht durch eine positiv geladene
Auffängeranode, die sich gegenüber der Photokathode befindet. Trifft nun ein Elektron auf die Auffängeranode, wird ein Strom
gemessen. In Abbildung \ref{fig:Anordnung1} ist dieses Prinzip veranschaulicht. In Abbildung \ref{fig:Photozelle} ist
die verwendete Photozelle skizziert, in welcher sich sowohl Anode als auch Kathode befinden. Die Kathode besteht dabei aus einer dünnen
Metallschicht im Inneren der Photozelle. Die Anode ist durch einen Drahtring wenige Millimeter vor der Kathode realisiert. Das Innere des
Glaskolbens, in welchem sich die Elektroden befinden, ist weitestgehend evakuiert, um Störeffekte mit Gasmolekülen zu vermeiden.

\begin{figure}
    \centering
    \begin{subfigure}[b]{0.45\linewidth}
        \centering
        \includegraphics[width=\textwidth]{pictures/Anordnung1.png}
        \caption{Schematischer Versuchsaufbau. \cite{AP01}}
        \label{fig:Anordnung1}
    \end{subfigure}
    \hspace{.1\linewidth}% Abstand zwischen Bilder
    \begin{subfigure}[b]{0.3\linewidth}
        \centering
        \includegraphics[width=\textwidth]{pictures/Photozelle.png}
        \caption{Skizze der Photozelle. \cite{AP01}}
        \label{fig:Photozelle}
    \end{subfigure}
\end{figure}

\noindent
Zwischen Kathode und Anode wird ein elektrisches Bremsfeld erzeugt, indem eine variable Spannungsquelle angeschlossen wird. Zur Messung der
ausgelösten Elektronen wird ein Picoamperemeter verwendet. Die elektrische Schaltung ist in Abbildung \ref{fig:Schaltung} dargestellt.
\begin{figure}
    \centering
    \includegraphics[scale=0.3]{pictures/ElektischeSchaltung.png}
    \caption{Elektrisches Schaltbild der Apparatur. \cite{AP01}}
    \label{fig:Schaltung}
\end{figure}

\noindent
Durch das Bremsfeld mit der Spannung $U_B$ können nur die Elektronen die Auffängeranode erreichen, deren Energie $E_{kin}>e_0U_B$ erfüllt.
Für die Grenzspannung
\begin{equation}
    e_0=\frac{1}{2}m_0v_{max}^2
\end{equation}
verschwindet der gemessene Strom, da selbst die Energie der schnellsten Elektronen nicht ausreicht, um die Anode zu erreichen. Nach
Gleichung \eqref{eqn:bilanz} gilt für diese Elektronen
\begin{equation}
    h\nu=\Phi_K+e_0U_g\;.
\end{equation}
Die Elektronen, die eine Geschwindigkeit $v<v_{max}$ besitzen, erreichen schon für eine Spannung $U_B<U_g$ nicht mehr die Anode. Deswegen
sinkt der gemessene Photostrom kontinuierlich ab, bis er bei $U_g$ Null erreicht. Dies ist in Abbildung \ref{fig:Photostrom} in einem
($U_B$-$I_P$)-Diagramm dargestellt.

\begin{figure}
    \centering
    \includegraphics[scale=0.3]{pictures/Photostrom.png}
    \caption{Photostrom $I_{P}$ in Abhängigkeit der Bremsspannung $U_B$. \cite{AP01}}
    \label{fig:Photostrom}
\end{figure}

\noindent
Die unterschiedlichen Geschwindigkeiten der ausgelösten Elektronen resultieren aus der bereits im Metall vorhandenen kinetischen Energie der
Elektronen, welche einer Fermi-Dirac-Verteilung folgt.
\\
\\
\noindent
Unter bestimmten Vorraussetzungen lässt sich zwischen dem gemessenen Strom $I_{P}$ und der Bremsspannung $U_B$ der Zusammenhang
\begin{equation*}
    I_{P}\propto U^2_g
\end{equation*}
zeigen.
\\
\\\noindent
Für den Fall, dass zwar $\Phi_K<h\nu$ gilt, also Elektronen aus der Kathode gelöst werden, aber für die Austrittsarbeit der Anode
$\Phi_A>h\nu$ gilt, wird kein Strom an der Anode gemessen. Dies liegt an dem elektrischen Gegenfeld, welches durch die Differenz der
Fermi-Niveaus entsteht und die Elektronen so stark abbremst, dass sie die Anode nicht mehr erreichen.
Mit einem beschleunigenden Potential $U_b$ kann jedoch wieder ein Strom gemessen werden, sobald
\begin{equation}
    h\nu+e_0U_{b}\geq \Phi_A
    \label{eqn:ausgleichb}
\end{equation}
gilt.
